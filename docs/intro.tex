\ssr{ВВЕДЕНИЕ}

Распределитель памяти SLAB основан на алгоритме, предложенном Джеффом Бонвиком для операционной системы SunOS~\cite{slab_info}.
Распределитель Бонвика строится вокруг объекта кэширования~\cite{slab_info}.
Внутри ядра значительное количество памяти выделяется на ограниченный набор объектов, например, дескрипторы файлов и другие общие структурные элементы.
Бонвик основывался на том, что количество времени, необходимое для инициализации регулярного объекта в ядре, превышает количество времени, необходимое для его выделения и освобождения~\cite{slab_info}.
Его идея состояла в том, что вместо того, чтобы возвращать освободившуюся память в общий фонд, оставлять эту память в проинициализированном состоянии для использования в тех же целях~\cite{slab_info}.
Например, если память выделена для mutex, функцию инициализации mutex необходимо выполнить только один раз, когда память впервые выделяется для mutex.
Последующие распределения памяти не требуют выполнения инициализации, поскольку она уже имеет нужный статус от предыдущего освобождения и обращения к деконструктору.

В Linux распределитель slab использует эти и другие идеи для создания распределителя памяти, который будет эффективно использовать пространство, и время~\cite{slab_info}.
Может возникнуть необходимость исследовать потребление памяти, выделяемой slab, процессом для контроля ее использования.
Существующий интерфейс, предоставляемый /proc/slabinfo, а также приложением slabtop, позволяет оценить общий размер кэшэй slab, но не позволяет отследить использование памяти конкретным процессом.

Целью работы является разработка загружаемого модуля ядра для мониторинга использования SLAB-кэша процессами в операционной системе Linux.

Задачи работы:
\begin{itemize}
	\item анализ и выбор методов и средств реализации загружаемого модуля ядра;
	\item разработка структур и алгоритмов, необходимых для работы загружаемого модуля ядра;
	\item анализ результатов работы разработанного загружаемого модуля ядра.
\end{itemize}

\clearpage
