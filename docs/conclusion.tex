\ssr{ЗАКЛЮЧЕНИЕ}

В рамках курсовой работы был разработан загружаемый модуль ядра для мониторинга использования SLAB процессамиа в операционной системе Linux.

Была поставлена задача по разработке загружаемого модуля ядра для отслеживания использования SLAB-кэша процессами в операционной системе Linux.
Был проведен анализ распределителей SLAB и SLUB и API для работы с ними, механизмы перехвата функций kprobes и ftrace, виртуальной файловой системы proc. Для перехвата функции был выбран фреймворк ftrace, поскольку он позволяет перехватит любую функцию по ее имени, загружается в ядро динамически и имеет задокументированный API.

Были разработаны последовательности преобразований в загружаемом модуле ядре для мониторинга использования SLAB-кэша процессами, алгоритмы загрузки и выгрузки разрабатываемого загружаемого модуля ядра, алгоритмы перехвата функции, чтения и записи для файла в виртуальной файловой системе proc и алгоритмы необходимых подменяемых функций.

Были разработаны реализации алгоритмов загрузки и выгрузки разрабатываемого загружаемого модуля ядра, алгоритмы перехвата функции, чтения и записи для файла в виртуальной файловой системе proc и алгоритмы необходимых подменяемых функций.
Был разработан make-файл для сборки загружаемого модуля ядра.

Была проведена проверка работы разработанного загружаемого модуля ядра для мониторинга использования SLAB-кэша процессами в операционной системе Linux.
Реализованный загружаемый модуль ядра работает исправно.

Были выполнены следующие задачи:
\begin{itemize}
	\item анализ и выбор методов и средств реализации загружаемого модуля ядра;
	\item разработка структур и алгоритмов, необходимых для работы загружаемого модуля ядра;
	\item анализ результатов работы разработанного загружаемого модуля ядра.
\end{itemize}
