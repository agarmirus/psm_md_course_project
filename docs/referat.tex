\begin{center}
	\LARGE\bfseries{РЕФЕРАТ}
\end{center}

Расчетно-пояснительная записка \pageref{LastPage} с., \totalfigures{} рис., \totaltables{} таблицы, 27 источников, 2 приложения.

ТЕПЛОВИЗОР, MOBIR~2T, СМАРТФОН, ANDROID, ВИДЕОПОТОК, USB-ТРАФИК.

Цель работы~--- создание метода получения видеопотока от мобильного тепловизора MobIR~2T.

В рамках научно-исследовательской работы был разработан метод получения видеопотока от мобильного тепловизора MobIR~2T.

Был проведен анализ предметной области, связанной с тепловизором MobIR~2T, и интерфейса USB. Были формализованы бизнес-правила для получения видеопотока от мобильного тепловизора. Была формализована задача и возможные способы получения видеопотока от мобильного тепловизора MobIR~2T.

Был проведен анализ информации, передаваемой тепловизором к хосту, в результате которого были опеределены формат и разрешение видеопотока.
Были предприняты неудачные попытки анализа USB-трафика, передаваемого между устройством и смартфоном, с помощью эмуляторов, виртуальных машин, аппаратных анализаторов и электронной вычислительной машины, работающей в качетстве посредника при передаче данных.
Был проведен сравнительный анализ и выбор программ для трансляции и записи экрана телефона на компьютере.
Была разработана пошаговая интсрукция по развертыванию приложений для получения видеопотока от мобильного тепловизора посредством осущетвления видеозаписи экрана смартфона.

В исследовательской части было проведено измерение среднего времени задержки при передаче видеопотока от мобильного тепловизора компьютеру.
Согласно полученным результатам, среднее время задержки приблизительно равна 139~мс.
